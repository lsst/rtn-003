\section{Introduction}\label{sec:intro}


The current model for LSST is that is provides proprietary data to approved users in Chile and the US. The data access model accommodates this restricted data rights policy. This policy requires control over access, publication, and sharing of proprietary data which any international data access center (IDAC) would have to comply with just as the US and Chile DACs do.

Access to LSST data products for any users will be possible through a Data Access Center (DAC). The United States's DAC will be hosted at the US Data Facility\footnote{currently National Center for Supercomputing Applications (NCSA)},
where registered LSST users will perform scientific queries. Most users will have access to a default set of resources at the DAC sufficient for basic queries and analysis. Users who require more resources will be able to apply for them, and those granted additional resources will be allowed (for example) to perform analysis on the full data releases using the LSST Science Platform (LSP). The LSP is documented with the vision given in \citeds{LSE-319}, with more formal requirements in \citeds{LDM-554} and the design in \citeds{LDM-542}. The Chilean DAC will be equivalent in functionality to the US DAC, but scaled-down in terms of the computational resources available for query and analysis given the smaller Chilean community \citedsp{LDM-572}.

%This document proposes a set of guidelines and policies for partner institutions -- in the US, Chile, or one of the International Contributors with signed Memoranda of Agreement -- that are interested in hosting the LSST data, in whole or in part, for their affiliated members as an independent Data Access Center (IDAC).
The following sections include the types of data products that could be hosted (Section \ref{sec:data}), the requirements and responsibilities that would be expected of an IDAC hosting LSST proprietary data products (Section \ref{sec:reqs}), and a description of the main costs {\it vs.} their science impacts (Section \ref{sec:costs}).

The contents of this draft document are meant to provide a preliminary resource for partner institutions who may be assessing the feasibility of hosting an IDAC. The specific mechanisms and processes by which future IDACs will negotiate the bulk transfer of data, the installation of software, etc. is considered beyond the scope of this document. A simplified checklist is given in \appref{sec:checklist}.

To better understand the sizes of LSST data products, \tabref{tab:sizes} gives an overview.


\begin{table}
\caption{Size summary based on \citeds{LDM-141} \label{tab:sizes}}

\begin{tabular}{l r l r r r }
\hline
 \bf{Table}  &    \bf{Bytes/row} &\bf{Rows (DR1 -> DR11)} &\bf{DR1 (TB)} &\bf{ $\times$ Growth } &\bf{DR10 (PB)}\\
\hline
 Object\_Lite    &1840   &$2.26^{10} -> 4.74^{10}$ &42  &2.1  &0.08 \\
 Object\_Extra   &20393  &$2.26^{10} -> 4.74^{10}$ &461 &2.1  &0.9  \\
 Source          &453    &$4.51^{11} -> 9.01^{12}$ &204  &20.0 &4.0  \\
 ForcedSrc       &41     &$1.20^{12} -> 5.01^{13}$ &49  &42  &2.0  \\
 DiaObject       &1405   &$7.94^{08} -> 1.54^{10}$  &1.1  &19.4 &0.02  \\
 DiaSource       &417    &$2.26^{09} -> 4.52^{10}$  &0.9&20 &0.02 \\
 DiaForcedSource &49     &$1.50^{10} -> 3.01^{11}$  &0.7 &20  &0.01 \\
\hline
 \multicolumn{6}{l}{Year ~1   raw images:$ 3 PB$, tables:$\sim 1 PB$, half for Object\_Extra,$ 0.2 PB$  Sources}\\
 \multicolumn{6}{l}{Year 10   raw images:$ 30 PB$, tables:$\sim 7 PB$,$ 4 PB$  Sources,$ 2.0 PB$  Forced ,$ 1 PB$  Object\_Extra}\\
\hline
\end{tabular}
\end{table}


All access to, and use of the LSST data and data products is subject to the policies described in \citeds{LDO-13}.

In addition to the sizes shown in \tabref{tab:sizes} it is interesting to consider how much access and potentially how much science
there is per table. This is discussed in detail in \citeds{PSTN-003}. The AMCL made an interesting table concerning this topic
which is reproduced here in \tabref{tab:use}. Feedback on the correctness of this table has been sought from PST.
\input{images/usetab.tex} % this is in the images git repo

