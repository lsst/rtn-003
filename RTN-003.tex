\documentclass[DM,authoryear,toc]{lsstdoc}
% lsstdoc documentation: https://lsst-texmf.lsst.io/lsstdoc.html
\input{meta}

% Package imports go here.

% Local commands go here.

%If you want glossaries
%\input{aglossary.tex}
%\makeglossaries

\title{Guidelines for Rubin Independent Data Access Centers}

% Optional subtitle
% \setDocSubtitle{A subtitle}

\author{%
Leanne Guy
}

\setDocRef{RTN-003}
\setDocUpstreamLocation{\url{https://github.com/rubin-observatory/rtn-003}}

\date{\vcsDate}

% Optional: name of the document's curator
% \setDocCurator{The Curator of this Document}

\setDocAbstract{%
This document provides guidelines for groups that are independent from the LSST Project and Operations (i.e. LSST Data Facility) and would like to stand up an independent Data Access Center (IDAC; existing data centers that could serve LSST data products are considered IDACs for purposes of this document). Some IDACs may want to serve only a subset of the LSST data products: this document proposes three portion sizes, from full releases to a “light” catalog without posteriors. Guidelines and requirements for IDACs in terms of data storage, computational resources, dedicated personnel, and user authentication are described, as well as a preliminary assessment of the cost impacts. Some institutions, even those inside the US and Chile, may serve LSST data products locally to their research community. Requirements and responsibilities for such institutional bulk data transfers are also described here. The purpose of this draft document is to serve as a preliminary resource for partner institutions wishing to assess the feasibility of hosting an IDAC.
}

% Change history defined here.
% Order: oldest first.
% Fields: VERSION, DATE, DESCRIPTION, OWNER NAME.
% See LPM-51 for version number policy.
\setDocChangeRecord{%
  \addtohist{1}{YYYY-MM-DD}{Unreleased.}{Leanne Guy}
}


\begin{document}

% Create the title page.
\maketitle
% Frequently for a technote we do not want a title page  uncomment this to remove the title page and changelog.
% use \mkshorttitle to remove the extra pages

% ADD CONTENT HERE
% You can also use the \input command to include several content files.

\appendix
% Include all the relevant bib files.
% https://lsst-texmf.lsst.io/lsstdoc.html#bibliographies
\section{References} \label{sec:bib}
\renewcommand{\refname}{} % Suppress default Bibliography section
\bibliography{local,lsst,lsst-dm,refs_ads,refs,books}

% Make sure lsst-texmf/bin/generateAcronyms.py is in your path
\section{Acronyms} \label{sec:acronyms}
\addtocounter{table}{-1}
\begin{longtable}{p{0.145\textwidth}p{0.8\textwidth}}\hline
\textbf{Acronym} & \textbf{Description}  \\\hline

AMCL & AURA Management Council for LSST \\\hline
ASDC & ASI Science Data Center (Italy) \\\hline
CADC & Canadian Astronomy Data Centre \\\hline
CAOM & Common Archive Observation Model \\\hline
CDN & Content Delivery Network \\\hline
CDS & Centre de Donnes astronomiques de Strasbourg \\\hline
CERN & European Organization for Nuclear Research \\\hline
CPU & Central Processing Unit \\\hline
DAC & Data Access Center \\\hline
DAX & Data Access Services \\\hline
DM & Data Management \\\hline
DMTN & DM Technical Note \\\hline
DR1 & Data Release 1 \\\hline
EPO & Education and Public Outreach \\\hline
ESAC & European Space Astronomy Centre \\\hline
ESNet & Energy Sciences Network \\\hline
FTE & Full-Time Equivalent \\\hline
GAVO & German Astronomical Virtual Observatory \\\hline
GB & Gigabyte \\\hline
Gb & Gigabit \\\hline
HEASARC & NASA's Archive of Data on Energetic Phenomena \\\hline
HEP &  High Energy Physics \\\hline
HIPS & Hierarchical Progressive Survey \\\hline
IBM & International Business Machines \\\hline
IDAC & Independent Data Access Center \\\hline
IN2P3 & Institut National de Physique Nucléaire et de Physique des Particules \\\hline
IP & Internet Protocol \\\hline
IPAC & No longer an acronym; science and data center at Caltech \\\hline
IVOA & International Virtual-Observatory Alliance \\\hline
IoA & Institute of Astronomy (Cambridge; also denoted IOA) \\\hline
LDF & LSST Data Facility \\\hline
LDM & LSST Data Management (Document Handle) \\\hline
LDO & LSST Document Operations (Document Handle) \\\hline
LPM & LSST Project Management (Document Handle) \\\hline
LSE & LSST Systems Engineering (Document Handle) \\\hline
LSP & LSST Science Platform \\\hline
LSST & Legacy Survey of Space and Time (formerly Large Synoptic Survey Telescope) \\\hline
MAST & Mikulski Archive for Space Telescopes \\\hline
MPA & Max Planck Institute for Astrophysics \\\hline
NAOJ & National Astronomical Observatory of Japan \\\hline
NCSA & National Center for Supercomputing Applications \\\hline
NED & NASA/IPAC Extragalactic Database \\\hline
NOAO & National Optical Astronomy Observatories (USA) \\\hline
PB & PetaByte \\\hline
PI & Principle Investigator \\\hline
PST & Project Science Team \\\hline
PSTN & Project Science Technical Note \\\hline
SAO & Smithsonian Astrophysical Observatory \\\hline
SDSS & Sloan Digital Sky Survey \\\hline
TAP & Table Access Protocol \\\hline
TB & TeraByte \\\hline
US & United States \\\hline
\end{longtable}

% If you want glossary uncomment below -- comment out the two lines above
%\printglossaries





\end{document}
